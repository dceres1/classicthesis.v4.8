%********************************************************************
% Appendix
%*******************************************************
% If problems with the headers: get headings in appendix etc. right
%\markboth{\spacedlowsmallcaps{Appendix}}{\spacedlowsmallcaps{Appendix}}
\chapter{Appendix}\label{sec:appendix}
\section{Configuration Files}
YAML files are the standard method for setting up configuration variables in the Nuga project. A collection of configuration examples is provided in this appendix.

\begin{lstlisting}[language=yaml, frame=tb, caption={Weed Infestation World config example}, label={lst:world-yaml}, float=h]
    seed: 2
    quadrant_size: [2.0, 2.0]
    quadrants:
      1:
        direction: x
        weed_density: 0.4   # weeds/m2
        spatial_distribution: clustered
        std_dev: [0.5, 0.5]
        outside_workspace: false
      2:
        direction: x
        weed_density: 1.2   # weeds/m2
        spatial_distribution: uniform
        outside_workspace: false
\end{lstlisting}    

\begin{lstlisting}[language=yaml, frame=tb, caption={ROS$2$ config example}, label={lst:ros2_control-yaml}, float=htb]
  # ROS2 Control
  controller_manager:
    ros__parameters:
      use_sim_time: True
      update_rate: 20 # Hz
  
      joint_state_broadcaster:
        type: joint_state_broadcaster/JointStateBroadcaster
  
      forward_position_controller_front:
        type: forward_command_controller/ForwardCommandController
      
      forward_position_controller_back:
        type: forward_command_controller/ForwardCommandController
  forward_position_controller_front:
    ros__parameters:
      joints:
        - front_x_axis_joint
        - front_y_axis_joint
        - front_implement_tool_joint
      interface_name: position
  forward_position_controller_back:
    ros__parameters:
      joints:
        - back_x_axis_joint
        - back_y_axis_joint
        - back_implement_tool_joint
      interface_name: position
\end{lstlisting} 

\clearpage
\section{Algorithms}

\begin{algorithm}[H]
\caption{Depth-First Search}
\label{alg:dfs}
\begin{algorithmic}[1]
\REQUIRE $root\_node$, $last\_node$
\ENSURE Explored Graph

\STATE $\text{visited} \gets \{\text{last\_node}\}$ \COMMENT{Initialize with last node}
\STATE $\text{stack} \gets [\text{root\_node}]$ \COMMENT{Start with root node}
\STATE $i \gets 0$ \COMMENT{Iteration counter}

\WHILE{stack is not empty}
    \STATE $i \gets i + 1$
    \STATE $\text{node} \gets \text{stack.pop}()$
    \IF{node $\notin$ visited}
        \STATE $\text{visited.add(node)}$
        \STATE $\text{children} \gets \text{get\_children\_nodes(node)}$
        
        \IF{children $\neq$ None}
            \STATE $\text{stack.extend(children)}$ \COMMENT{Add children to stack}
        \ENDIF
    \ENDIF
    \IF{$i == 5000$}
        \STATE \textbf{break} \COMMENT{Early termination}
    \ENDIF
\ENDWHILE
\end{algorithmic}
\end{algorithm}


\begin{algorithm}[H]
  \caption{Dijkstra's Algorithm}
  \label{alg:dijkstra}
  \begin{algorithmic}[1]
  \REQUIRE $graph$, $start\_node$, $goal\_node$
  \ENSURE Shortest path from $start\_node$ to $goal\_node$
  
  \STATE $cost \gets \text{defaultcost}(\infty)$ \COMMENT{Distance from start to each node}
  \STATE $prev \gets \text{dict()}$ \COMMENT{To reconstruct shortest path}
  \STATE $cost[start\_node] \gets 0$
  \STATE $queue \gets \text{priority queue initialized with } (0, start\_node)$
  
  \WHILE{queue is not empty}
      \STATE $(current\_cost, current\_node) \gets \text{queue.pop()}$
      \IF{$current\_node == goal\_node$}
          \STATE \textbf{break}
      \ENDIF
      \FORALL{$neighbor$ of $current\_node$ in $graph$}
          \STATE $new\_cost \gets current\_cost + \text{weight}(current\_node, neighbor)$
          \IF{$new\_cost < cost[neighbor]$}
              \STATE $cost[neighbor] \gets new\_cost$
              \STATE $prev[neighbor] \gets current\_node$
              \STATE $\text{queue.push}(new\_cost, neighbor)$
          \ENDIF
      \ENDFOR
  \ENDWHILE
  
  \STATE \textbf{return} shortest path and cost from $start\_node$ to $goal\_node$
  \end{algorithmic}
\end{algorithm}
  

\begin{algorithm}[H]
    \caption{Get Next Stop}
    \label{alg:get-next-stop}
    \begin{algorithmic}[1]
    \REQUIRE Detections, Robot pose
    \ENSURE Next stop pose and assigned tasks

    \IF{$\texttt{stop\_pose} \neq \texttt{None}$}
        \STATE \textbf{return}
    \ENDIF

    \IF{\texttt{detections} is empty}
        \STATE $\texttt{nuga\_state} \gets \texttt{"moving"}$
        \STATE \textbf{return}
    \ENDIF

    \STATE $(\texttt{tasks\_in}, \texttt{tasks\_out}) \gets \texttt{get\_tasks\_in\_future\_path()}$
    \STATE $\texttt{now} \gets \texttt{current time}$

    \IF{\texttt{tasks\_out} is not empty}
        \FORALL{$(\texttt{idx}, \texttt{pose}) \in \texttt{tasks\_out}$}
            \STATE $\texttt{tasks\_out\_ws[idx]} \gets \texttt{pose}$
            \STATE \texttt{add\_marker("map", now, "out", idx, pose, "orange")}
        \ENDFOR
    \ENDIF

    \IF{\texttt{tasks\_in} is empty}
        \STATE $\texttt{nuga\_state} \gets \texttt{"moving"}$
        \STATE \textbf{return}
    \ENDIF

    \STATE \texttt{algorithm.set\_robot\_pose(nuga\_pose)}
    \STATE \texttt{algorithm.set\_tasks(tasks\_in)}
    \STATE $(\texttt{stop\_pose}, \texttt{stop\_tasks}) \gets \texttt{algorithm.compute\_next\_stop()}$
    \STATE \texttt{add\_marker("map", now, "goal", stops\_counter + 1, stop\_pose, "blue")}

    \STATE \texttt{clear\_polygons()}
    \STATE $id \gets 0$
    \FORALL{$(\texttt{name}, \texttt{imp}) \in \texttt{implements}$}
        \STATE \texttt{add\_polygon("map", now, name + "\_ws", id, imp.future\_ws\_pose)}
        \STATE $id \gets id + 1$
    \ENDFOR
    \end{algorithmic}
\end{algorithm}



\section{Another Appendix Section Test}
Equidem detraxit cu nam, vix eu delenit periculis. Eos ut vero
constituto, no vidit propriae complectitur sea. Diceret nonummy in
has, no qui eligendi recteque consetetur. Mel eu dictas suscipiantur,
et sed placerat oporteat. At ipsum electram mei, ad aeque atomorum
mea. There is also a useless Pascal listing below: \autoref{lst:useless}.

\begin{lstlisting}[float=b,language=Pascal,frame=tb,caption={A floating example (\texttt{listings} manual)},label=lst:useless]
for i:=maxint downto 0 do
begin
{ do nothing }
end;
\end{lstlisting}

%Ei solet nemore consectetuer nam. Ad eam porro impetus, te choro omnes
%evertitur mel. Molestie conclusionemque vel at, no qui omittam
%expetenda efficiendi. Eu quo nobis offendit, verterem scriptorem ne
%vix.

