%************************************************
\chapter{Conclusions}\label{ch:conclusions}
%************************************************
This thesis presents a comprehensive comparison between task allocation algorithms: heuristic, graph search, optimization-based, and market-based. Different paradigms were used to approach the problem of assigning detected weeds to onboard tools in the NUGA agricultural robotic platform. A simulation framework was developed to benchmark the performance of these algorithms across varying weed density scenarios, allowing for a detailed analysis of mission duration and tool idle time.

Results showed that both the graph search and optimization-based methods significantly outperform the baseline approach, particularly in medium to high-density scenarios. Notably, the optimization-based algorithm demonstrated strong scalability, maintaining acceptable computation times for larger problem sizes and reducing tool idle time by up to 94.9\%. These findings highlight the potential of intelligent task allocation strategies to substantially improve the efficiency and productivity of autonomous field operations.

Investigating the impact of increasing the number of tools in the robot, analyzing productivity trade-offs and exploring globally optimal solutions by extending the robot's sensing capabilities are promising avenues for future research that could be explore using the current simulation.

This thesis demonstrates that intelligent task allocation strategies can significantly enhance the efficiency and productivity of field robotic systems, paving the way for more sustainable and scalable solutions in agriculture.

%*****************************************
%*****************************************
%*****************************************
%*****************************************
%*****************************************